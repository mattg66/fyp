\chapter{Introduction}
\label{chap:intro}

\section{The Client}
\label{intro:client}

The client is CX Labs UK within Cisco Systems. CX Labs provides lab space for
use by business units internal to the company. Most of the space is used for
the testing of customer networks by SVS (Solution Validation Services). SVS
provides bespoke testing services to customers wishing to use Cisco’s expertise
to test a range of situations, from regression and firmware testing to full
upgrade and migration plans.\newline
To match the customer's environment as close as possible, a scaled-down
version of the customer's network is usually recreated in the lab space managed
by CX Labs. CX Labs hold many devices that cover most of the Cisco portfolio,
which allows for the recreation of most networks. Most of this lab space is
hosted within the internal Cisco corporate network, which requires any users to
be employees of Cisco to access testbeds. More and more customers
however are requesting remote access to their testbeds. To facilitate this, a
fully isolated DMZ environment is provided, which allows direct WAN
connectivity to a testbed, allowing for a VPN tunnel to be established and
hence remote access granted to a testbed from any location to any permitted
person.

\subsubsection{Current Infrastructure}
The current infrastructure to support the testbeds consists of four Nexus 9K core switches, with \gls{fex}s to provide copper connectivity. Currently, the N9Ks do not have any form of \gls{vpc} configured, and just use \gls{stp} for redundancy. Top-of-rack connectivity is provided by Catalyst switches. Each project has a dedicated VLAN to isolate communications between different projects, and a virtual router and services stack is hosted on a vCenter environment to provide internet and \gls{vpn} connectivity to the project. Each project also has several terminal servers attached to its VLAN so that the console ports of the testbed devices can be easily accessed in case of a device problem.

\subsubsection{Current Project Pipeline}
When a new testbed build is requested, the project topology is reviewed, which will indicate how much rackspace will be required to accommodate the project. A new VLAN for the project will then be manually created across all of the associated networking equipment, including the core N9Ks, and the top-of-rack switches that live in the selected racks. Terminal servers will also have to be reconfigured so that they have the correct subnet configured and the correct sub-interface configured. The next step is to provision a virtual CSR1000v router hosted on vCenter that will provide \gls{nat} and internet access to the newly created VLAN. A services stack will then be deployed to take care of remote access \gls{vpn} and other associated services. Both of these steps require the manual addition of a \gls{dpg} to the \gls{dvs} present in vCenter.
\section{The Problem}
\label{intro:problem}

Whilst the existing
solution is functional and works, the major downside is that there is no
automation or configuration management solution deployed. This means over time,
configuration drift occurs as manual changes are made, but not made the same to
all switches. Unused VLANs are also not removed from the switches which leads
to bloat and a larger configuration than is required. The same situation also
occurs on vCenter with the configuration of the DPGs within the DVS where
unused DPGs are never removed or labeled incorrectly.\newline
The manual configuration of the required routing, switching and VPN termination as well as
other lab requirements such as NTP, DNS and RADIUS all take a lot of time to
configure. This time could be better utilised, such as preparing the physical
rack space for the racking and stacking of new equipment. Terminal server configuration also requires manual intervention, as a previous project may have used a different IP address space.

With this large amount of manual configuration, mistakes are easily made which results in extra time spent troubleshooting, and the potential for existing projects to be affected by accidental configuration mistakes. There is also no easy way to view all configurations present on the networking devices without manually connecting to each device and inspecting the running configuration.
\section{Aims and Objectives}
\label{intro:aims}

This project aims to provide a solution that will automate the
configuration of the DMZ network infrastructure, as well as the configuration of the associated project infrastructure, such as the project router, services stack and terminal servers. This will be achieved by providing a web-based dashboard that allows the user to provision a new project, which will then automatically configure the required infrastructure. The solution will revolve around the rackspace being virtually created inside the web UI, so that projects can be allocated racks and have the associated infrastructure that is tied to a rack automatically provisioned. The solution will also provide a view of the current utilisation of the lab space by projects, as well as the current utilisation of the lab space as a whole.

The solution will use Cisco \gls{aci} to provide network connectivity. This removes the complexities that would otherwise be associated with provisioning many network devices via \gls{ssh} or RESTCONF. \gls{aci} will also make the network highly scalable, with the ability to add additional leafs and \gls{fex}s to support any expansion of the rackspace. VMWare vCenter will also be used to host virtual machines associated with project infrastructure, such as the project's virtual router and services stack.

This report will also detail the design and conception of a testbed that will be used to test the automation platform against \gls{aci} and vCenter. This design can then be used to build a fully-fledged infrastructure to support testing in the future.
\subsection{Deliverables}
\label{intro:aims:deliverables}

\begin{itemize}
    \item A web-based dashboard that allows the user to manage and provision projects
    \item User guides
          \begin{itemize}
              \item User guide for the dashboard
              \item User guide for how to provision ACI, vCenter and other associated network infrastructure so that it will be compatible with the automation solution
          \end{itemize}
    \item A report detailing the design and implementation of the solution
\end{itemize}

\section{Constraints}
\label{intro:constraints}

Because the testbed utilised for testing and development is in a remote datacenter, physical access to the testbed will be limited.
This may result in delays if a problem with the physical infrastructure occurs which could lead to the project running over time.
The solution will be designed to be as resilient as possible, but there is a chance that the solution may not be able to recover from a failure.
There may also be a chance that access to the testbed is revoked, which would result in the project being delayed until access is restored.

External factors such as the availability of the testbed's internet uplink and the power supply may also have an impact on the delivery of the project within the required timescale.
