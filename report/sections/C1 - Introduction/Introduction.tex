\chapter{Introduction}
\label{chap:intro}

\section{The Client}
\label{intro:client}

The client is CX Labs UK within Cisco Systems. CX Labs provides lab space for
use by business units internal to the company. Most of the space is used for
the testing of customer networks by SVS (Solution Validation Services). SVS
provide bespoke testing services to customers wishing to use Cisco’s expertise
to test a range of situations, from regression and firmware testing to full
upgrade and migration plans.\newline
In order to match the customers environment as close as possible, a scaled down
version of the customers network is usually recreated in the lab space managed
by CX Labs. CX Labs hold many devices that cover most of the Cisco portfolio,
which allows for the recreation of most networks. Most of this lab space is
hosted within the internal Cisco corporate network, which requires any users to
be employees of Cisco in order to access testbeds. More and more customers
however are requesting remote access to their testbeds. To facilitate this, a
fully isolated DMZ environment is provided, which allows direct WAN
connectivity to a testbed, allowing for a VPN tunnel to be established and
hence remote access granted to a testbed from any location to any permitted
person.

\section{The Problem}
\label{intro:problem}

Currently the solution for the DMZ network infrastructure consists of 4 Nexus
9K devices, with FEXs for RJ45 connectivity and 2960S ToR switches. Whilst this
solution is functional and works, the major downside is that there is no
automation or configuration management solution deployed. This means over time,
configuration drift occurs as manual changes are made, but not made the same to
all switches. Unused VLANs are also not removed from the switches which leads
to bloat and a larger configuration than is required. The same situation also
occurs on vCenter with the configuration of the DPGs within the DVS where
unused DPGs are never removed or are labelled incorrectly.\newline
The manual configuration of the required routing and VPN termination as well as
other lab requirements such as a NTP, DNS and RADIUS all take a lot of time to
configure. This time could be better utilised, such as preparing the physical
rack space for the racking and stacking of new equipment.

\section{Aims and Objectives}
\label{intro:aims}

The aim of this project is to provide an easy to use dashboard that allows
CX Labs to easily onboard new projects and manage the existing projects. This
will be achieved by providing a web based interface that allows the user to
create a new project, which will then automatically create the required
configuration within ACI, vCenter, and other associated network infrastructure.
This will also allow the user to easily manage the existing projects, such as
expanding a projects rackspace utilisation, or removing a project from the
environment. The dashboard will also provide a view of the current projects
utilisation of the lab space, and the current utilisation of the lab space as a
whole. 
