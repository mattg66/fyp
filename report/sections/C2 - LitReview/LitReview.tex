\chapter{Literature Review}
\label{chap:litreview}
The aim of this literature review is to research and analyse existing
solutions, documentation and research on the automation of networking
infrastructure. To ensure this review is of maximum usefulness, it will also
involve analysing best practises and standards when developing software and
automation solutions. This will allow for the optimisation of the planning and
implementation stages of the project that will subsequently follow.

Sources for this review will be from relevant books, online websites,
professional publications and Request for Comments. Multiple services will be
used to identify and obtain sources that can be used for the purposes of this
review.
\begin{itemize}
    \item Google Scholar
    \item IEEE Explore
    \item ResearchGate
    \item University of Portsmouth Library
\end{itemize}
Research was conducted that was related to the following set of topics:
\begin{itemize}
    \item What is software defined networking?
    \item What types of software defined networking exist?
    \item What are the advantages and disadvantages?
\end{itemize}

\section{Definition of Software Defined Networks}
\label{litreview:definition}

All industry experts and academics define software defined networking
similarly, that is providing automation and intelligence to networks via the
means of software and APIs.~\citet{11} state that ``SDN was originally coined
to represent the ideas and work around OpenFlow at Stanford University''.\\ SDN
is often defined using four pillars, \citet{11} define these as the following:
\begin{enumerate}
    \item ``The control and data planes are decoupled. Control functionality is
          removed from network devices that will become simple (packet)
          forwarding
          elements''
    \item ``Forwarding decisions are flow based, instead of destination based.
          A flow is broadly defined by a set of packet field values acting as a
          match
          (filter) criterion and a set of actions (instructions)''
    \item ``Control logic is moved to an external entity, theso-called SDN
          controller''
    \item ``The network is programmable through software applications running
          on top of the NOS that in-teracts with the underlying data plane
          devices.''
\end{enumerate}

\section{Overview of Software Defined Networks}
\label{litreview:overview}

Since this project is centred around software defined networking and the
automation of these networks, it is critical to ensure that the principles and
their method of operations are understood.\\
An official survey paper from the IEEE that analysed the state of SDN provides
a good explanation as to why the need for network programmability arose in the
first place. Conventionally, “Computer networks are typically built from a
large number of network devices such as routers, switches and numerous types of
middleboxes”~\cite{1}. \citet{1} go on to state that due to the large
amount of manual configuration required to achieve the desired traffic flow,
“network management and performance tuning is quite challenging and thus
error-prone”.
This leads to one of the solutions, Software Defined Networking. “Software
Defined Networking (SDN) is a new networking paradigm which the forwarding
hardware is decoupled from control decisions. It promises to dramatically
simplify network management and enable innovation and evolution”. Software
defined networking “is designed to use standardized application programming
interfaces (APIs) to quickly allow network programmers to define and
reconfigure the way data or resources are handled within a network.” \cite{9}

\section{Types of Software Defined Networks}
\label{litreview:types}

Software Defined Networking is a blanket term for any solution that tackles
conventional networking with a programmatic view. There are ``the two main
delivery models: Imperative and declarative'' \citet{10}. \citet{10} go on to
state that Imperative SDN is
where ``A centralized controller (typically a clustered set of controllers)
functions as the network’s ‘brain’'' and that declarative SDN is where
``the intelligence is distributed out to the network fabric. While policy is
centralized, policy enforcement isn’t''. \citet{10} give ``a protocol such as
OpenFlow explicitly
telling network switches precisely what to do and how to do it'' to be an
example of imperative SDN.