\chapter{Literature Review}
\label{chap:litreview}
The aim of this literature review is to research and analyse existing
solutions, documentation and research on the automation of networking
infrastructure. To ensure this review is of maximum usefulness, it will also
involve analysing best practises and standards when developing software and
automation solutions. This will allow for the optimisation of the planning and
implementation stages of the project that will subsequently follow.

Sources for this review will be from relevant books, online websites,
professional publications and Request for Comments. Multiple services will be
used to identify and obtain sources that can be used for the purposes of this
review.
\begin{itemize}
    \item Google Scholar
    \item IEEE Explore
    \item ResearchGate
    \item University of Portsmouth Library
\end{itemize}
Research was conducted that was related to the following set of topics:
\begin{itemize}
    \item What is software defined networking?
    \item What types of software defined networking exist?
    \item What are the advantages and disadvantages?
    \item What automation is currently used in the networking world?
    \item What automation solutions exist in the industry?
\end{itemize}

\section{Definition of Software Defined Networks}
\label{litreview:definition}

All industry experts and academics define software defined networking
similarly, that is providing automation and intelligence to networks via the
means of software and APIs.~\citet{11} state that ``SDN was originally coined
to represent the ideas and work around OpenFlow at Stanford University''.\\ OpenFlow
is often defined using four pillars, \citet{11} define these as the following:
\begin{enumerate}
    \item ``The control and data planes are decoupled. Control functionality is
          removed from network devices that will become simple (packet)
          forwarding
          elements''
    \item ``Forwarding decisions are flow based, instead of destination based.
          A flow is broadly defined by a set of packet field values acting as a
          match
          (filter) criterion and a set of actions (instructions)''
    \item ``Control logic is moved to an external entity, theso-called SDN
          controller''
    \item ``The network is programmable through software applications running
          on top of the NOS that in-teracts with the underlying data plane
          devices.''
\end{enumerate}
These four pillars outline the overall function and principles of OpenFlow. Whilst this sounds good, OpenFlow has many disadvantages that have lead to it become less favourable when deploying larger networks. As a network switch/router has to submit a flow request to the controller for every unkown source/destination contained in a packet, a large demand on the controller will occr. ``Sending a flow request (Packet-In message) to the SDN-Controller for each unknown packet will confuse the SDN-Controller because the SDN-Controller has to compute the forwarding rules for each new packet and then install it to the flow tables in all the data forwarding nodes (SDN-Switches)'' \citep{app11156999}. \citeauthor{app11156999} go on to say that ``This high volume of traffic and computational overhead will cause SDN-Controller overhead and increase the time it takes for flow rules to be placed, affecting network efficiency and scalability''. This makes OpenFlow unsuitable for larger networks, as for every new flow, a lookup must occur which results in higher latency and high system demands when it comes to forwarding packets. This has led to the term 'Software Defined Networking' being used as a more general term, as it is not limited to the use of OpenFlow, but for any solution that tackles conventional networking with a programmatic view. There are ``the two main
delivery models: Imperative and declarative'' \citet{10}. \citet{10} go on to
state that Imperative SDN is
where ``A centralized controller (typically a clustered set of controllers)
functions as the network’s ‘brain’'' and that declarative SDN is where
``the intelligence is distributed out to the network fabric. While policy is
centralized, policy enforcement isn’t''. \citet{10} give ``a protocol such as
OpenFlow explicitly
telling network switches precisely what to do and how to do it'' to be an
example of imperative SDN.

In summary, \gls{sdn} is a blanket term for a variety of methods of delivery intelligence to conventionally statically configured networks.

\section{Declarative vs Imperative SDN}
\label{litreview:declarativevsimperative}
As briefly touched upon in the previous section, \gls{sdn} is broken up into two main types, imperative and declarative. This section will explore the differences between the two in more detail. The imperative and declarative definitions originate from software development. The ``Imperative paradigm can be viewed as a traditional programming structure and allows the programmer to specify all the steps to solve any particular problem. In the declarative paradigm, one only needs to specify what program must do, not how to do it'' \citep{LATIF2020102563}. Translated into networking, imperative SDN perfectly explains what OpenFlow set out to do, and that is for the controller to make the decision and inform the end networking device exactly how to forward and handle the packet. Declarative solutions such as OpFlex are designed for ``transferring abstract policy from a modern network controller to a set of smart devices capable of rendering policy'' \cite{bhardwaj_2020}. \citeauthor{bhardwaj_2020} goes on to state how ``OpFlex is designed to work as part of a declarative control system such as Cisco ACI in which abstract policy can be shared on demand.''


\section{Why Use Software Defined Networks}
\label{litreview:overview}

Since this project is centered around \gls{sdn} and the
automation of these networks, it is critical to ensure that the principles and
their method of operations are understood, and the benefits of using it.\\
An official survey paper from the IEEE that analysed the state of SDN provides
a good explanation as to why the need for network programmability arose in the
first place. Conventionally, “Computer networks are typically built from a
large number of network devices such as routers, switches and numerous types of
middleboxes”~\cite{1}. \citet{1} go on to state that due to the large
amount of manual configuration required to achieve the desired traffic flow,
“network management and performance tuning is quite challenging and thus
error-prone”.
This leads to one of the solutions, Software Defined Networking. “Software
Defined Networking (SDN) is a new networking paradigm which the forwarding
hardware is decoupled from control decisions. It promises to dramatically
simplify network management and enable innovation and evolution”. Software
defined networking “is designed to use standardized application programming
interfaces (APIs) to quickly allow network programmers to define and
reconfigure the way data or resources are handled within a network.” \cite{9}

\section{Software Defined Networking Solutions}
\label{litreview:types}
As expected, many manufacturers have released and developed solutions that use the principles of \gls{sdn} to automate network operations using a variety of hardware. This section will explore the different types of \gls{sdn} solutions that are available in the industry.

\subsubsection{Cisco ACI}
Cisco \gls{aci} is a proprietary solution from Cisco that uses the Nexus 9000 series of switches using a special firmware version. Whilst the design of the fabric remains essentially unchanged, with spine-and-leaf being the required design, where \gls{aci} does introduce change is with how configuration and policy are applied to the networking devices. ``In a leaf-spine ACI fabric, Cisco is provisioning a native Layer 3 IP fabric that supports equal-cost multi-path (ECMP) routing between any two endpoints in the network, but uses overlay protocols, such as virtual extensible local area network (VXLAN) under the covers to allow any workload to exist anywhere in the network'' \citep{duffy2014cisco}. \gls{aci} also features plug-and-play fabric discovery, where new switches are automatically discovered by the controller and can be onboarded with ease, making future network expansion very easy to achieve.

\subsubsection{Juniper Apstra}
``Juniper’s Apstra solution provides
a deployment method called connectivity templates, which allow administrators to create and reuse validated templates to set up multi-vendor networks. It supports multiple device operation systems, including Cisco NX-OS, Nvidia Cumulus and Juniper Junos OS'' \citep{9914530}. The main advantage over Cisco \gls{aci} is the fact that it supports multiple vendors so you are not locked to just one manufacturer's current and future hardware. Apstra is still in its infancy, as it was only released in December 2020 \citep{9914530}, which means that documentation and training material is still sparse, and it has not been proven in the field to be as reliable in a mission-critical environment as Cisco \gls{aci}.