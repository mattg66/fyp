\chapter{Literature Review}
\label{chap:litreview}
The aim of this literature
review is to research and analyse existing
solutions, documentation and
research on the automation of networking
infrastructure. To ensure this review
is of maximum usefulness to the development of the solution, it will also
involve analysing best practices and
standards when developing software and
automation solutions. This will allow
for the optimisation of the planning and
implementation stages of the project
that will subsequently follow.

Sources
for this review will be relevant
books, online websites,
professional
publications and Request for Comments.
Multiple services will be
used to
identify and obtain sources that can be used
for this review.
\begin{itemize}

      \item Google Scholar
      \item IEEE Explore
      \item ResearchGate
      \item University of Portsmouth Library
\end{itemize}
The research performed was related to the following set of
topics:
\begin{itemize}
      \item What is software defined networking?
      \item What types of software defined networking exist?
      \item What are the advantages and disadvantages?
      \item What automation is currently used in the networking world?
      \item What automation solutions exist in the industry?
\end{itemize}

\section{Definition of Software Defined Networks}
\label{litreview:definition}

All industry experts and academics define
software defined networking
similarly, that is providing automation and
intelligence to networks via the
means of software and APIs.~\citet{11} state
that ``SDN was originally coined
to represent the ideas and work around
OpenFlow at Stanford University''.\\
Four pillars are often used to define the differences between conventional
networking and OpenFlow. \citet{11} define these as the
following:
\begin{enumerate}
      \item The decoupling of the control and data planes
      \item Forwarding decisions are based on flows, which represent a set of packets with the same characteristics
      \item Decision-making logic is moved to a centralised controller which has visibility over the whole network
      \item Providing the ability to programmatically interact with the network through the use of \gls{api}s and \gls{sdk}s.
\end{enumerate}

Whilst these four pillars illustrate positive aspects of OpenFlow, it has many scalability disadvantages that have led it to become less favourable when deploying larger networks \citep{8784036}. This has resulted in additional
solutions being developed. \gls{sdn} has been divided into two subcategories, imperative and declarative \citep{10}.Imperative \gls{sdn} is where ``A centralized controller (typically a clustered set of controllers) functions as
the network’s ‘brain’'' \citep{10} and that declarative \gls{sdn} is where ``the intelligence is distributed out to the network fabric. While policy is centralized, policy enforcement isn’t'' \citep{10}. Using this definition,
OpenFlow can be placed into the imperative \gls{sdn} category, as the controller is used to directly influence the packet forwarding process \citep{11}. With \gls{sdn} now referring to different methods of making networks smart, a concrete definition has become harder to reach. \gls{sdn} is referred to as ``an innovative architecture that separates the control plane from the data plane to simplify and speed up the management of large networks'' \citep{app11156999}, however, other researchers take a more programmatic view of \gls{sdn}. \citeauthor{9} defines \gls{sdn} as  ``a new networking architecture that is designed to use standardized application programming interfaces (APIs) to quickly allow network programmers to define and reconfigure the way data or resources are handled within a network'' \citep{9}. Whilst these are two different perspectives, modern \gls{sdn} is a combination of both, with both programmable \gls{api}s and a decoupled control and data plane both being features of \gls{sdn}.

In summary, a \gls{sdn} is a network architecture that separates the control plane from the data plane and provides automation and intelligence to networks through software and \gls{api}s, whilst providing a centralised point of administration to the network administrator.

% There
% are ``the
% two main
% delivery
% models: Imperative and declarative''
% \citet{10}.
% \citet{10}
% go on to
% state that
% Imperative SDN is
% where ``A
% centralized
% controller
% (typically a clustered set
% of controllers)
% functions as
% the
% network’s ‘brain’''
% and that declarative SDN
% is where
% ``the intelligence is
% distributed out to the
% network fabric. While
% policy is
% centralized, policy
% enforcement isn’t''. Using
% this definition,
% OpenFlow can be placed into the
% imperative \gls{sdn} category,
% as the
% controller is used to directly influence
% the packet forwarding process.

% These
% four pillars outline
% the
% overall
% function
% and principles of OpenFlow.
% Whilst
% this sounds good,
% OpenFlow
% has many
% disadvantages that have led to it
% becoming
% less favourable
% when
% deploying
% larger networks. As a network
% switch/router has
% to submit a flow
% request to the
% controller for every unknown
% source/destination
% contained in a
% packet, a large
% demand on the controller will
% occur. ``Sending a
% flow request
% (Packet-In
% message) to the SDN-Controller for
% each unknown packet
% will confuse
% the
% SDN-Controller because the SDN-Controller
% has to compute the
% forwarding
% rules
% for each new packet and then install it to
% the flow tables in
% all the
% data
% forwarding nodes (SDN-Switches)''
% \citep{app11156999}. With this
% additional
% lookup stage that does not occur in
% regular networks, high load can
% be placed
% on the controller and associated
% networking infrastructure, and this
% is before
% any packet forwarding has
% actually occurred. This makes OpenFlow
% unsuitable for
% larger networks, as for
% every new flow, a lookup must occur
% which results in
% higher latency and high
% system demands when it comes to
% forwarding packets.
% This has led to the term
% 'Software Defined Networking'
% being used as a more
% general term, as it is not
% limited to the use of OpenFlow,
% but to any solution
% that tackles conventional
% networking with a programmatic
% view. There are ``the
% two main
% delivery models:
% Imperative and declarative''
% \citet{10}. \citet{10}
% go on to
% state that
% Imperative SDN is
% where ``A
% centralized controller
% (typically a clustered set
% of controllers)
% functions as
% the network’s ‘brain’''
% and that declarative SDN
% is where
% ``the intelligence is
% distributed out to the
% network fabric. While
% policy is
% centralized, policy
% enforcement isn’t''. Using
% this definition,
% OpenFlow can be placed into the
% imperative \gls{sdn} category,
% as the
% controller is used to directly influence
% the packet forwarding process.

% In
% summary, \gls{sdn} is a blanket term for a
% variety of methods of delivering
% intelligence and reporting to conventionally
% statically configured networks.
% Software
% defined networking “is designed to use
% standardized application
% programming
% interfaces (APIs) to quickly allow network
% programmers to define
% and
% reconfigure the way data or resources are handled
% within a network.”
% \citep{9}

\section{Declarative vs Imperative SDN}
\label{litreview:declarativevsimperative}
As briefly touched upon in the
previous section, \gls{sdn} is broken up into two main types, imperative and
declarative. The imperative and declarative definitions originally originate from
software development. Imperative programming is the traditional programming
method, where the programmer specifies all steps in order to achieve a desired
outcome \citep{LATIF2020102563}. Declarative programming, however, is where the overall goal is
specified, instead of all of the intermediary steps that must be completed to
achieve the goal  \citep{LATIF2020102563}. Translated into networking,
imperative SDN perfectly explains what OpenFlow set out to do, and that is for
the controller to make the decision and inform the end networking device
exactly how to forward and handle the packet. Since the creation of OpenFlow,
many declarative solutions have been created to provide a more scalable
solution. Declarative solutions such as OpFlex are designed for ``transferring
abstract policy from a modern network controller to a set of smart devices
capable of rendering policy'' \citep{bhardwaj_2020}. \citeauthor{bhardwaj_2020}
goes on to state how ``OpFlex is designed to work as part of a declarative
control system such as Cisco ACI in which abstract policy can be shared on
demand.'' Another declarative protocol is NETCONF, which allows for device
configuration to be read and modified through the use of Remote Procedure Call
\citep{LATIF2020102563}.

\section{Why Use Software Defined Networks}
\label{litreview:overview}

Since this project is centered around \gls{sdn} and
the
automation of these networks, it is critical to ensure that the principles
and
their method of operations are understood, and the benefits of using it.\\
An official survey paper from the IEEE that analysed the state of SDN provides
a good explanation as to why the need for network programmability arose in the
first place. Conventionally, ``Computer networks are typically built from a
large number of network devices such as routers, switches and numerous types of
middleboxes'' \citep{1}. \citeauthor{1} goes on to state that due to the large
amount of manual configuration required to achieve the desired traffic flow,
“network management and performance tuning is quite challenging and thus
error-prone”.  Having a centralised controller allows for a single point of management \citep{7785187}, which makes the management of a large network much easier, as is found with \gls{sdn} environments.

As referenced earlier, the ability to programmatically control and interact with a network is also a key benefit of \gls{sdn}. \gls{api}s ``quickly allow network programmers to define and reconfigure the way data or resources are handled within a network'' \citep{9}. A northbound \gls{api} provides a ``high-level \gls{api} between the controller and the applications'' \citep{6844664} which need to interact with the network.
\section{Software Defined Networking Solutions}
\label{litreview:types}
As expected, many manufacturers have released and
developed solutions that use the principles of \gls{sdn} to automate network
operations using a variety of hardware. This section will explore the different
types of \gls{sdn} solutions that are available in the industry.

\subsubsection{Cisco ACI}
Cisco \gls{aci} is a proprietary solution from Cisco
that uses the Nexus 9000 series of switches using a special firmware version.
Whilst the design of the datacenter fabric remains essentially unchanged, with
spine-and-leaf being the required design, where \gls{aci} does introduce change
is with how configuration and policy are applied to the networking devices.
Cisco ACI still utilises the leaf-spine architecture which has been proven to
be highly scalable and provide the data throughput that is required for modern
datacenters \citep{7}. The whole fabric is Layer 3 so that \gls{ecmp} can be utilised to
share load across multiple links, however, overlay protocols such as
\gls{vxlan} are utilised to allow any workload to exist at any point in the
fabric \citep{duffy2014cisco}.
\gls{aci} also features plug-and-play fabric
discovery, where new switches are automatically discovered by the controller
and can be onboarded with ease, making future network expansion very easy to
achieve.

\subsubsection{Juniper Apstra}
``Juniper’s Apstra solution provides
a
deployment method called connectivity templates, which allow administrators to
create and reuse validated templates to set up multi-vendor networks. It
supports multiple device operation systems, including Cisco NX-OS, Nvidia
Cumulus and Juniper Junos OS'' \citep{9914530}. The main advantage Apstra has over Cisco
\gls{aci} is the fact that it supports multiple vendors. This prevents becoming locked in with a vendor's future and current hardware portfolio. Apstra is still in its
infancy, as it was only released in December 2020 \citep{9914530}, which means
that documentation and training material are still sparse, and it has not been
proven in the field to be as reliable in a mission-critical environment as
Cisco \gls{aci}.

\subsubsection{VMWare NSX}
VMWare NSX is a software-defined
networking solution that is designed to be used in a virtualised environment.
Whilst this provides many advantages for improving networking when using
virtual machines and applications, NSX provides no management for physical
networking, and is purely focused on providing automation and networking for
applications. It mainly provides tools and telemetry for day-two operations and
troubleshooting \citep{2}. This means that whilst NSX is a good solution for
virtualised environments, it is not suitable for physical networking
deployments.

\section{Software Defined Networking Alternatives}
\label{litreview:alternatives}
Whilst \gls{sdn} is a great
solution for
automating networks, it is not the only solution. This section
will explore the
alternatives to \gls{sdn} and their advantages and
disadvantages.
\subsubsection{Ansible}
Ansible is an open-source piece of software from Red
Hat that is described as a configuration management tool \citep{4}, where a form of state
description can be written, and then verified through the use of Ansible
\citep{powerofansible}. Whilst Ansible does not provide the same level of
automation as \gls{sdn}, it does provide a way to automate the configuration of
network devices, and can be used to automate the deployment of new devices. The
network configuration must be designed and built before Ansible is used to push
configuration to the devices. Ansible does help solve issues of scale and
complexity, as it can be used to push configuration to multiple devices at
once, and can be used to automate the deployment of new devices.

\subsubsection{Chef}
Chef is another open-source configuration management tool
that is similar to Ansible. Chef can easily handle up to 10000 nodes from a
single chef server \citep{sabharwal2014automation}, however, it is designed to
use an agent which must be installed upon the device to be automated which adds
another layer of complexity. It also requires the design of configuration and
is useful only for deployment and preventing configuration drift.

\section{Developing Software To Interface With SDN}
\label{litreview:developing}
\gls{sdn} already provides the facilities to
programmatically interact with the network through the use of \gls{rest}
\gls{api}s. Most solutions provide a 'northbound' \gls{api} which can be used
by developers to interact with the policy defined in the \gls{sdn} controller.
Any changes made via this \gls{api} will then be propagated down to the network
devices, with no interaction with the actual devices themselves required. The
northbound \gls{api} allows developers to focus on controlling the \gls{sdn}
instead of worrying about sending actual commands down to the network devices
\citep{7899569}. There is no standard to northbound \gls{api}s and they are
heavily vendor specific \citep{7502469}, this means that an application is only
able to support one \gls{sdn} platform, unless compatibility for multiple is
explicitly developed and accounted for. This means that the correct platform
should be chosen in advance of a solution being developed.

\section{Disadvantages of SDN}
\label{litreview:disadvantages}
Whilst the benefits of \gls{sdn} have been discussed, it is not without its shortcomings. One of the big issues with \gls{sdn} is choosing a solution, as vendor inter-operability is still a significant drawback of \gls{sdn} \citep{5}. Once a solution is selected, it is very difficult to migrate to another or add hardware from another vendor. Security is also hard to find as part of direct integration with most \gls{sdn} solutions, which results in multiple control planes, one for data and one for security \citep{5}. This results in multiple points of administration and a lack of consistency in configuration style and integration between the networking and security elements. \gls{sdn} also relies on centralised controllers, whilst these controllers are not critical for traffic flow in declarative settings, they remain a centralised point of failure that could lead to downtime \citep{rana2019software}. This centralised nature also ties into making the controller a target for \gls{ddos} attacks, as taking out the controller would result in disruption to wider network operation \citep{7289347}.

\section{Conclusion}
\label{litreview:conclusion}
Software Defined Networking uses a modern approach to
improve the efficiency and scalability of conventional networking. Through the
use of centralised control and administration, network administrators no longer
have to develop such advanced scripts or have as large a workforce to manage
and maintain large networks, that can often have thousands of devices.
Declarative \gls{sdn} solutions seem to be the way forward, as they take all of
the advantages of having centralised control and administration, but don't
place as heavy a load on the centralised controller when it comes to
influencing packet flow and routing decisions. The ability to easily develop
software to control network connectivity, without any of the fuss of
integrating and managing any devices that provide physical connectivity is a
big bonus and will be imperative to the success of the project. As the solution
is highly specific to a testing environment, there are currently no alternative
solutions that provide the exact feature set that is desired.