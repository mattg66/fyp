\chapter{Conclusion}
\label{chap:conclusion}
The overall goal of this automation platform was to reduce the amount of human interaction required when a new project enters a testing environment. A user can now attach the solution to an \gls{aci} fabric and vCenter environment with the required prerequisites, and deploy projects to rackspace with no manual configuration being required. Whilst this has been met, manual intervention from the user is still required as no remote access \gls{vpn} is provisioned automatically.
The user interface was designed in a simple manner that shouldn't require extensive training to utilise, and only basic familiarisation with how the solution works would be required.

The objective of having the user experience revolve around the rackspace was successful, as now only the rackspace has to be considered when deploying a new project. 


\subsection{Future Expansion}
\label{sec:future-expansion}
The solution has been built in a modular fashion, so in the future,  any potential expansion will be easy to achieve. Features that were left out due to time restrictions such as a login system and continuous monitoring would be useful to implement in the future. Deployment of a services stack to provide \gls{vpn} and \gls{dns} would also be very useful and is a high priority for future development.

\subsection{Learning Points}
\label{sec:learning-points}
If this project was to be restarted, several mistakes that were made could be avoided to improve the development experience. Making use of Kanban more effectively, and even tying in with GitHub so that commits can be associated with jobs on the board would be advantageous. This is because project development performance can be easily viewed, and future features can be prioritised and have the correct amount of time allocated.
