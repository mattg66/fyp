\chapter{Requirements}
\label{chap:requirements}

This chapter will provide in more detail the requirements for the automation
platform. The requirements will be split into 2 sections, namely the functional
requirements and non-functional requirements. The functional requirements
will outline the features that the platform must have and the non-functional
requirements will outline the requirements that the platform must meet to be successful.
To prioritise the requirements, the MoSCoW method will be used. This method will allow for the requirements to be prioritised and will ensure that
the most important requirements are met first. The MoSCoW method is a
prioritisation method that splits requirements into 4 categories, namely Must
Have, Should Have, Could Have and Won’t Have.
\section{Functional Requirements}
Functional requirements are outlined by the IEEE as a function that a system or a system component must be able to perform. \citep{159342}

\begin{center}
    \begin{table}[H]
        \begin{tabular}{l p{0.6\linewidth} l}
            \hline
            \textbf{ID} & \textbf{Details}
                        & \textbf{Priority}
            \\ \hline
            FR1         & Visual representation of rack space
                        & Must Have
            \\ \hline
            FR2         & Add and remove racks from the space
                        & Must Have
            \\ \hline
            FR3         & Add and remove Terminal Servers from racks
                        & Must Have
            \\ \hline
            FR4         & Add and remove Fabric Nodes from racks
                        & Must Have
            \\ \hline
            FR5         & Add, remove and update projects
                        & Must Have
            \\ \hline
            FR6         & Expand or contract a projects consumption of rack
            space       & Must Have
            \\ \hline
            FR7         & Automate configuration of ACI fabric
                        & Must Have
            \\ \hline
            FR8         & Deploy virtual router using vCenter API
                        & Must Have
            \\ \hline
            FR9         & Deploy virtual services stack to provide remote
            access VPN  & Could
            Have
            \\ \hline
            FR10        & Continuous monitoring of ACI and vCenter health
                        & Won’t Have
            \\
            \hline
            FR11        & Terminal server automated management
                        & Must Have
            \\ \hline
            FR12        & Login system to restrict access                   & 
            Could Have                                                         \\
        \end{tabular}
        \caption{Functional Requirements}
        \label{requirements:functional}
    \end{table}
\end{center}

FR1 - FR2 outlines the requirements to have the rack space visualised in the web application of the solution. The idea behind this is that the application will simplify the process of adding and removing racks which will allow for the rack space to be easily recreated, and will also allow for the rack space to be easily updated if the rack space changes.
It will also help show the utilisation of the space, and allow for project planning to be carried out more easily.

FR3 - FR4 outlines the ability to associate \gls{aci} nodes and terminal servers to racks, this is required so that the automation backend can push the required config out when a rack is onboarded into a project. This also adds the ability to add and remove nodes and terminal servers if any physical changes occur in the rack space.

FR5 details the ability of the automation platform to store projects. This will provide the core functionality of the platform, where the current projects are stored and managed through the automation platform.

FR6 details the requirement to expand and contract a project's rack space utilisation, this will allow for the project to be scaled up or down as required which is a common occurrence.

FR7 outlines the core automation functionality of the platform. This is to automate the deployment of connectivity to the \gls{aci} fabric based on the selected rack space and associated fabric nodes.

FR8 outlines the deployment of a virtual router to the vCenter automation platform. This will provide internet connectivity to the project network created by FR7.

FR9 provides the ability to automate the creation of a project services stack, this may include services such as \gls{vpn} and \gls{ntp} to name a few.

FR10 outlines the possibility of having continuous status monitoring of ACI and vCenter, however, due to the required time to implement this feature it has been marked as a Won’t Have.

FR11 details the ability to also automate the terminal servers associated with racks which will ensure that terminal servers are connected to projects upon their onboarding.

FR12 details the possibility of implementing a login system, whilst this would be a useful feature and should be implemented at some point, the project will be hosted on a secure network that requires access to be granted, so the login system may be out of scope given the time restrictions.
\section{Non-Functional Requirements}
Non-functional requirements describe the non-behavioral characteristics of a system, capturing the properties and constraints under which a system must operate \citep{12} 
\begin{center}
    \begin{table}[H]
        \begin{tabular}{l p{0.6\linewidth} l}
            \hline
            \textbf{ID}             & \textbf{Details}
                                    & \textbf{Priority}
            \\ \hline
            NFR1                    & Must be easy to use for staff with less technical
            knowledge
                                    & Must Have
            \\ \hline
            NFR2                    & The system status should be easily visible to staff (e.g.
            errors, project status) & Must Have                                                                                       \\ \hline
            NFR3                    & The system should be able to easily integrate with existing \gls{aci} fabric deployments & Could have
        \end{tabular}
        \caption{Non-Functional Requirements}
        \label{requirements:non-functional}
    \end{table}
\end{center}

NFR1 outlines the requirement for the web application to be easy to use for less experienced team members. Through the use of abstraction, the networking and configuration can be hidden behind an easy to use web interface through the use of automation scripts that are run as a result of the user's actions.

NFR2 shows that the system must report the status to staff via the use of status indicators. This should show the progress of the automation scripts as they progress through automating and applying the configuration to various elements of the network.

NFR3 outlines for the platform to be able to integrate with existing \gls{aci} fabric deployments. This will allow for the platform to be used in a production environment without the need to reconfigure and rearchitect the fabric. Ideally, the platform should be deployed alongside a new fabric in a greenfield deployment.