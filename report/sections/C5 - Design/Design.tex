\chapter{Design}
\label{chap:design}
As the project has utilised an iterative approach, the design has changed and been refined throughout the project. The design of the software will be broken down into 3 sections, namely the design of the web application and the design of the ACI and vCenter configuration.

\section{Web Application}
\label{design:web-application}
\subsection{Architecture}
\label{design:web-application:architecture}

A client-server architecture will be utilised to provide the user-facing experience, and the automation and data handling logic hosted on the server. By using this model, many clients can request and interact with data that is hosted on one central server. The backend and frontend that make up the application will be separate from one another, and will therefore be developed independently, with the frontend interacting with the backend via a REST API. This allows for the front end to be a SPA, which facilitates a better user experience due to the lack of page refreshes upon every request.

The server will process all requests generated by the front end and also make requests to the various APIs that will be required to automate the network deployment. The database will also store all data required by the server to generate the appropriate network configuration that is required to automate the deployment of projects to the network.

\subsection{Frontend}
\label{design:web-application:frontend}
Next.js will be used to power the frontend of the application as it is an enhancement of React.js and provides server-side rendering and acceleration of pages. React uses a modular 'componentised' approach to building the frontend, which allows for the creation of reusable components that can be used throughout the application. This allows for the creation of a modular and scalable frontend that can be easily extended and maintained.

React.js also has an extensive library of open-source components and libraries that can be utilised to make developing the frontend easier and more feature complete. Due to the complexity of some required features, such as having a drag-and-drop interface for the recreation of the lab space, the use of a library such as React Flow will be required as the time required to develop such a feature would be out of the scope of this project.

\subsection{Backend}
\label{design:web-application:backend}
The backend of the application will be written in PHP, using the Laravel PHP framework. Laravel is a popular PHP framework that will accelerate the development process, as it features an inbuilt ORM, API routing system and an authentication system that can be implemented. Laravel utilises SQL-based databases, and as such MariaDB will be used as the database for the application. Laravel also features an in-built HTTP client which will be required to interact with the ACI and vCenter APIs which are all REST-based.

Laravel follows the MVC architecture, however as the frontend is a React SPA, Laravel will only be used to provide and consume the data via its REST API.
