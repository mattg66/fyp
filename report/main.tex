%%%%%%%%%%%%% You can ignore this stuff
\documentclass[bsc]{abdnthesis}
\usepackage[T1]{fontenc}
\usepackage[utf8]{inputenc}
\usepackage{booktabs}
\usepackage{longtable}
\usepackage{csquotes}

\usepackage[style=apa, backend=biber, natbib=true]{biblatex}
\DefineBibliographyStrings{english}{%
  bibliography = {References},
}

\addbibresource{refs.bib}

% This creates hyperlinks and moves the contents links to the page number for clarity
\usepackage[linktocpage=true,hidelinks]{hyperref}

\def\subsectionautorefname{Section}
\def\subsubsectionautorefname{Section}
\def\sectionautorefname{Section}
\def\chapterautorefname{Chapter}

%%%%%%%%%%%%%%%%% Space for your own packages here %%%%%%%%%%%%%%%%%%%%%%%%%%%%%%
% Uncomment for blank lines between paragraphs rather than
% indents
\usepackage[parfill]{parskip}
\usepackage{xcolor} %Great if you want coloured text
\usepackage{algorithm}
\usepackage{algorithmic} %Algorithm styles, need to be nested for the example shown
\usepackage{graphicx,color}
\usepackage{listings}

\hypersetup{linktocpage}
\hypersetup{
  linktoc=all,     %set to all if you want both sections and subsections linked
}

\definecolor{codegreen}{rgb}{0,0.6,0}
\definecolor{codegray}{rgb}{0.5,0.5,0.5}
\definecolor{codepurple}{rgb}{0.58,0,0.82}
\definecolor{backcolour}{rgb}{0.95,0.95,0.92}

\definecolor{lightgray}{rgb}{.9,.9,.9}
\definecolor{darkgray}{rgb}{.4,.4,.4}
\definecolor{purple}{rgb}{0.65, 0.12, 0.82}

\lstdefinelanguage{JavaScript}{
  keywords={typeof, new, true, false, catch, function, return, null, catch,
      switch, var, if, in, while, do, else, case, break},
  keywordstyle=\color{blue}\bfseries,
  ndkeywords={class, export, boolean, throw, implements, import, this},
  ndkeywordstyle=\color{darkgray}\bfseries,
  identifierstyle=\color{black},
  sensitive=false,
  comment=[l]{//},
  morecomment=[s]{/*}{*/},
  commentstyle=\color{purple}\ttfamily,
  stringstyle=\color{red}\ttfamily,
  morestring=[b]',
  morestring=[b]',
}

\lstdefinestyle{normal}{
  backgroundcolor=\color{backcolour},
  commentstyle=\color{codepurple},
  keywordstyle=\color{magenta},
  numberstyle=\tiny\color{codegray},
  stringstyle=\color{codegreen},
  basicstyle=\ttfamily\footnotesize,
  breakatwhitespace=false,
  breaklines=true,
  captionpos=b,
  keepspaces=true,
  numbers=left,
  numbersep=5pt,
  showspaces=false,
  showstringspaces=false,
  showtabs=false,
  tabsize=2
}
\lstdefinestyle{php}{
  backgroundcolor=\color{backcolour},
  commentstyle=\color{codepurple},
  keywordstyle=\color{magenta},
  numberstyle=\tiny\color{codegray},
  stringstyle=\color{codegreen},
  basicstyle=\ttfamily\footnotesize,
  otherkeywords={with},
  breakatwhitespace=false,
  breaklines=true,
  captionpos=b,
  keepspaces=true,
  numbers=left,
  numbersep=5pt,
  showspaces=false,
  showstringspaces=false,
  showtabs=false,
  tabsize=2
}

\lstdefinestyle{python}{
  backgroundcolor=\color{backcolour},
  commentstyle=\color{codepurple},
  keywordstyle=\color{magenta},
  numberstyle=\tiny\color{codegray},
  stringstyle=\color{codegreen},
  basicstyle=\ttfamily\footnotesize,
  otherkeywords={with},
  breakatwhitespace=false,
  breaklines=true,
  captionpos=b,
  keepspaces=true,
  numbers=left,
  numbersep=5pt,
  showspaces=false,
  showstringspaces=false,
  showtabs=false,
  tabsize=2
}

%%%%%%%%%%%%%%%%%%%%%%%%%%%%%%%%%%%%%%%%%%%%%%%%%%%%%%%%%%%%%%%%%%%%%%%%%%%%%%%%%

\title{A Web Driven SDN Orchestrator For The Provisioning of ACI Fabric and Lab
  Infrastructure}
\author{Matthew Gaynor}
\upnumber{922830}

\projecttype{Engineering Project}
\school{School of Computing}

%%%% In the final submission of a thesis, this should only be the year
%%%% of submission.  However, it is useful to use \date{\today} for drafts so that
%%%% they don't get mixed up.

\date{2023}

%% It is useful to split the document up as chapters and include
%% them.  LaTeX will sort out all the numbering and cross-referencing
%% for you --- if you run it enough times!

\begin{document}

%%%% Create the title page and standard declaration.

\maketitle
\makedeclaration

%%%% Then the abstract and acknowledgements

\section*{Abstract}
The purpose of this project is to provide a solution to CX Labs UK within
  Cisco Systems LTD that will streamline the operation of DMZ lab-space. The
  application shall allow lab staff to prepare network infrastructure for an
  incoming project with little to no CLI interaction and it shall be web based.
  ACI, VMware ESXi and vCenter shall be used to manage the networking and virtual
  machines respectively. The lab-space will be represented by a small testbed
  containing the minimum quantity of networking equipment and compute resources
  required to accurately simulate the environment.

\section*{Acknowledgements}
I would like to thank Cisco, my manager and good friend, Dave Smith, for his
close help in allowing me to use equipment necessary for testing and developing the solution.
\section*{Consent to Share}
I consent / do not consent for this project to be archived by the University
Library and potentially used as an example project for future students.
\newacronym{svs}{SVS}{Solution Validation Services}
\newacronym{cli}{CLI}{Command Line Interface}
\newacronym{cx}{CX}{Customer Experience}
\newacronym{dmz}{DMZ}{Demilitarised Zone}
\newacronym{wan}{WAN}{Wide Area Network}
\newacronym{vpn}{VPN}{Virtual Private Network}
\newacronym{nos}{NOS}{Network Operating System}
\newacronym{fex}{FEX}{Fabric Extender}
\newacronym{tor}{ToR}{Top of Rack}
\newacronym{dpg}{DPG}{Distrubited Port Group}
\newacronym{dvs}{DVS}{Distributed Virtual Switch}
\newacronym{ntp}{NTP}{Network Time Protocol}
\newacronym{dns}{DNS}{Domain Name System}
\newacronym{radius}{RADIUS}{Remote Authentication Dial-In User Service}
\newacronym{aci}{ACI}{Application Centric Infrastructure}
\newacronym{spa}{SPA}{Singple Page Application}
\newacronym{api}{API}{Application Programming Interface}
\newacronym{rest}{REST}{Representational State Transfer}
\newacronym{orm}{ORM}{Object Relational Mapping}
\newacronym{http}{HTTP}{HyperText Transfer Protocol}
\newacronym{mvc}{MVC}{Model View Controller}
\newacronym{erd}{ERD}{Entity Relationship Diagram}
\newacronym{apic}{APIC}{Application Policy Infrastructure Controller}
\newacronym{oob}{OOB}{Out of Band}
\newacronym{nat}{NAT}{Network Address Translation}
\newacronym{vm}{VM}{Virtual Machine}
\newacronym{vrf}{VRF}{Virtual Routing and Forwarding}
\newacronym{epg}{EPG}{Endpoint Group}
\newacronym{ddi}{DDI}{DNS, DHCP and IP Address Management (IPAM)}
\newacronym{bd}{BD}{Bridge Domain}
\newacronym{ospf}{OSPF}{Open Shortest Path First}
\newacronym{mp-bgp}{MP-BGP}{Multi-Protocol Border Gateway Protocol}
\newacronym{as}{AS}{Autonomous System}
\newacronym{ssh}{SSH}{Secure Shell}
\newacronym{vpc}{vPC}{Virtual Port Channel}
\newacronym{aaep}{AAEP}{Attachable Access Entity Profile}
\newacronym{lacp}{LACP}{Link Aggregation Control Protocol}
\newacronym{svi}{SVI}{Switch Virtual Interface}
\begingroup     
\let\clearpage\relax
\glsaddall
\printglossary[style=modsuper, type=\acronymtype, nonumberlist, nopostdot, title=Abbreviations] 
\endgroup

%% please indicate whether you are happy for other students to
%% view your project in the future

%%%% These are macros ( each macro is defined by \newcommand )
%%%% These macros make cross-referencing easier, building on the
%%%% already very useful \autoref{ref}.  You tweak them or build your own.

%% \autorefp{ref}
%% Like \autoref, but adds the page number
%% e.g. Section 3.2.4 (p345)
\newcommand{\autorefp}[1]{\autoref{#1} (p\pageref*{#1})}

%% \autorefnp{ref}
%% Like \autoref, but adds both the section name and page number
% e.g. Section 3.2.4 "blah blah" (p345)
\newcommand{\autorefnp}[1]{\autoref{#1} ``\nameref{#1}'', (p\pageref*{#1})}

%% \see{ref}
%% Handy shorthand for inserting a hyperlinked xref.
%%   e.g. (see Section 3.2.4, p134)
%%        (see Figure 5.1, p23)
%%        (see Table 4.7, p999)
\newcommand{\see}[1]{\hyperref[#1]{(see \autoref*{#1}, p\pageref*{#1})}}

%% \seenamed{ref}
%% Handy shorthand for inserting a hyperlinked xref that also includes
%% the name of the section being referred to.
%% (see Section 3.2.4 "blah blah", p134)
%% (see Figure 5.1 "blah blah blah!", p23)
%% (see Table 4.7 "blah blah blah blah blah blah", p999)
\newcommand{\seenamed}[1]{\hyperref[#1]{(see \autoref*{#1} ``\nameref*{#1}'',
    p.\pageref*{#1})}}

%% \bq Block Quotations.
\newcommand{\bq}[2][]{\singlespacing \begin{quote}
    \begin{small} ``\textit{#2}'' \end{small} #1 \end{quote} \doublespacing}

%% \qq{quotation}
%% Shortcut for doing inline quotations (with 6699 quotation marks around italic text)
%% e.g. \qq{To be, or not to be}
\newcommand{\qq}[1]{{\enquote{\textit{#1}}}}

%%%% It should have a table of contents, but delete the other two as
%%%% necessary.

\tableofcontents
\listoftables
\listoffigures

\chapter{Introduction}
\label{chap:intro}

\section{The Client}
\label{intro:client}

The client is CX Labs UK within Cisco Systems. CX Labs provides lab space for
use by business units internal to the company. Most of the space is used for
the testing of customer networks by SVS (Solution Validation Services). SVS
provide bespoke testing services to customers wishing to use Cisco’s expertise
to test a range of situations, from regression and firmware testing to full
upgrade and migration plans.\newline
In order to match the customers environment as close as possible, a scaled down
version of the customers network is usually recreated in the lab space managed
by CX Labs. CX Labs hold many devices that cover most of the Cisco portfolio,
which allows for the recreation of most networks. Most of this lab space is
hosted within the internal Cisco corporate network, which requires any users to
be employees of Cisco in order to access testbeds. More and more customers
however are requesting remote access to their testbeds. To facilitate this, a
fully isolated DMZ environment is provided, which allows direct WAN
connectivity to a testbed, allowing for a VPN tunnel to be established and
hence remote access granted to a testbed from any location to any permitted
person.

\section{The Problem}
\label{intro:problem}

Currently the solution for the DMZ network infrastructure consists of 4 Nexus
9K devices, with FEXs for RJ45 connectivity and 2960S ToR switches. Whilst this
solution is functional and works, the major downside is that there is no
automation or configuration management solution deployed. This means over time,
configuration drift occurs as manual changes are made, but not made the same to
all switches. Unused VLANs are also not removed from the switches which leads
to bloat and a larger configuration than is required. The same situation also
occurs on vCenter with the configuration of the DPGs within the DVS where
unused DPGs are never removed or are labelled incorrectly.\newline
The manual configuration of the required routing and VPN termination as well as
other lab requirements such as a NTP, DNS and RADIUS all take a lot of time to
configure. This time could be better utilised, such as preparing the physical
rack space for the racking and stacking of new equipment.

\section{Aims and Objectives}
\label{intro:aims}

The aim of this project is to provide an easy to use dashboard that allows
CX Labs to easily onboard new projects and manage the existing projects. This
will be achieved by providing a web based interface that allows the user to
create a new project, which will then automatically create the required
configuration within ACI, vCenter, and other associated network infrastructure.
This will also allow the user to easily manage the existing projects, such as
expanding a projects rackspace utilisation, or removing a project from the
environment. The dashboard will also provide a view of the current projects
utilisation of the lab space, and the current utilisation of the lab space as a
whole. 

\chapter{Literature Review}
\label{chap:litreview}
The aim of this literature
review is to research and analyse existing
solutions, documentation and
research on the automation of networking
infrastructure. To ensure this review
is of maximum usefulness, it will also
involve analysing best practises and
standards when developing software and
automation solutions. This will allow
for the optimisation of the planning and
implementation stages of the project
that will subsequently follow.

Sources for this review will be from relevant
books, online websites,
professional publications and Request for Comments.
Multiple services will be
used to identify and obtain sources that can be used
for the purposes of this
review.
\begin{itemize}
      \item Google Scholar

      \item IEEE Explore
      \item ResearchGate
      \item University of

            Portsmouth
            Library
\end{itemize}
Research was conducted that
was
related to the following
set of topics:
\begin{itemize}
      \item What is
            software defined networking?

      \item What types of software

            defined
            networking exist?
      \item What are the

            advantages and
            disadvantages?
      \item What automation is

            currently used in
            the
            networking world?

      \item
            What automation solutions exist in the

            industry?
\end{itemize}

\section{Definition of Software Defined Networks}
\label{litreview:definition}

All industry experts and academics define
software defined networking
similarly, that is providing automation and
intelligence to networks via the
means of software and APIs.~\citet{11} state
that ``SDN was originally coined
to represent the ideas and work around
OpenFlow at Stanford University''.\\
OpenFlow
is often defined using four
pillars, \citet{11} define these as the
following:
\begin{enumerate}

      \item
            The decoupling of the
            control and data planes

      \item
            Forwarding decisions

            based on flows, which
            represent a set of
            packets
            with
            the same

            characteristics
      \item
            Decision-making

            logic is moved to a centralised
            controller which
            has

            visibility over the whole network.
      \item Providing the

            ability to programmatically interact with the network through the use of

            \gls{api}s and \gls{sdk}s.
\end{enumerate}
These four pillars outline
the
overall function and principles of OpenFlow. Whilst this sounds good,
OpenFlow
has many disadvantages that have led to it becoming less favourable
when
deploying larger networks. As a network switch/router has to submit a flow
request to the controller for every unknown source/destination contained in a
packet, a large demand on the controller will occr. ``Sending a flow request
(Packet-In message) to the SDN-Controller for each unknown packet will confuse
the SDN-Controller because the SDN-Controller has to compute the forwarding
rules for each new packet and then install it to the flow tables in all the
data forwarding nodes (SDN-Switches)'' \citep{app11156999}. With this
additional lookup stage that does not occur in regular networks, high load can
be placed on the controller and associated networking infrastructure, and this
is before any packet forwarding has actually occurred. This makes OpenFlow
unsuitable for larger networks, as for every new flow, a lookup must occur
which results in higher latency and high system demands when it comes to
forwarding packets. This has led to the term 'Software Defined Networking'
being used as a more general term, as it is not limited to the use of OpenFlow,
but to any solution that tackles conventional networking with a programmatic
view. There are ``the two main
delivery models: Imperative and declarative''
\citet{10}. \citet{10} go on to
state that Imperative SDN is
where ``A
centralized controller (typically a clustered set of controllers)
functions as
the network’s ‘brain’'' and that declarative SDN is where
``the intelligence is
distributed out to the network fabric. While policy is
centralized, policy
enforcement isn’t''. Using this definition, OpenFlow can be placed into the
imperative \gls{sdn} category, as the controller is used to directly influence
the packet forwarding process.

In summary, \gls{sdn} is a blanket term for a
variety of methods of delivering intelligence and reporting to conventionally
statically configured networks. Software
defined networking “is designed to use
standardized application programming
interfaces (APIs) to quickly allow network
programmers to define and
reconfigure the way data or resources are handled
within a network.” \citep{9}

\section{Declarative vs Imperative SDN}
\label{litreview:declarativevsimperative}
As briefly touched upon in the
previous section, \gls{sdn} is broken up into two main types, imperative and
declarative. This section will explore the differences between the two in more
detail. The imperative and declarative definitions originally originate from
software development. Imperative programming is the traditional programming
method, where the programmer specifies all steps in order to achieve a desired
outcome. Declarative programming, however, is where the overall goal is
specified, instead of all of the intermediary steps that must be completed to
achieve the goal  \citep{LATIF2020102563}. Translated into networking,
imperative SDN perfectly explains what OpenFlow set out to do, and that is for
the controller to make the decision and inform the end networking device
exactly how to forward and handle the packet. Since the creation of OpenFlow,
many declarative solutions have been created to provide a more scalable
solution. Declarative solutions such as OpFlex are designed for ``transferring
abstract policy from a modern network controller to a set of smart devices
capable of rendering policy'' \citep{bhardwaj_2020}. \citeauthor{bhardwaj_2020}
goes on to state how ``OpFlex is designed to work as part of a declarative
control system such as Cisco ACI in which abstract policy can be shared on
demand.'' Another declarative protocol is NETCONF, which allows for device
configuration to be read and modified through the use of Remote Procedure Call
\citep{LATIF2020102563}.

\section{Why Use Software Defined Networks}
\label{litreview:overview}

Since this project is centered around \gls{sdn} and
the
automation of these networks, it is critical to ensure that the principles
and
their method of operations are understood, and the benefits of using it.\\
An official survey paper from the IEEE that analysed the state of SDN provides
a good explanation as to why the need for network programmability arose in the
first place. Conventionally, ``Computer networks are typically built from a
large number of network devices such as routers, switches and numerous types of
middleboxes'' \citep{1}. \citeauthor{1} goes on to state that due to the large
amount of manual configuration required to achieve the desired traffic flow,
“network management and performance tuning is quite challenging and thus
error-prone”.

\section{Software Defined Networking Solutions}
\label{litreview:types}
As expected, many manufacturers have released and
developed solutions that use the principles of \gls{sdn} to automate network
operations using a variety of hardware. This section will explore the different
types of \gls{sdn} solutions that are available in the industry.

\subsubsection{Cisco ACI}
Cisco \gls{aci} is a proprietary solution from Cisco
that uses the Nexus 9000 series of switches using a special firmware version.
Whilst the design of the fabric remains essentially unchanged, with
spine-and-leaf being the required design, where \gls{aci} does introduce change
is with how configuration and policy are applied to the networking devices.
Cisco ACI still utilises the leaf-spine architecture which has been proven to
be highly scalable and provide the data throughput that is required for modern
datacenters. The whole fabric is Layer 3 so that \gls{ecmp} can be utilised to
share load across multiple links, however, overlay protocols such as
\gls{vxlan} are utilised to allow any workload to exist at any point in the
fabric \citep{duffy2014cisco}.
\gls{aci} also features plug-and-play fabric
discovery, where new switches are automatically discovered by the controller
and can be onboarded with ease, making future network expansion very easy to
achieve.

\subsubsection{Juniper Apstra}
``Juniper’s Apstra solution provides
a
deployment method called connectivity templates, which allow administrators to
create and reuse validated templates to set up multi-vendor networks. It
supports multiple device operation systems, including Cisco NX-OS, Nvidia
Cumulus and Juniper Junos OS'' \citep{9914530}. The main advantage over Cisco
\gls{aci} is the fact that it supports multiple vendors so you are not locked
to just one manufacturer's current and future hardware. Apstra is still in its
infancy, as it was only released in December 2020 \citep{9914530}, which means
that documentation and training material is still sparse, and it has not been
proven in the field to be as reliable in a mission-critical environment as
Cisco \gls{aci}.

\subsubsection{VMWare NSX}
VMWare NSX is a software-defined
networking solution that is designed to be used in a virtualised environment.
Whilst this provides many advantages for improving networking when using
virtual machines and applications, NSX provides no management for physical
networking, and is purely focused on providing automation and networking for
applications. It mainly provides tools and telemetry for day-two operations and
troubleshooting \citep{2}. This means that whilst NSX is a good solution for
virtualised environments, it is not suitable for physical networking
deployments.

\section{Software Defined Networking Alternatives}
\label{litreview:alternatives}
Whilst \gls{sdn} is a great
solution for
automating networks, it is not the only solution. This section
will explore the
alternatives to \gls{sdn} and their advantages and
disadvantages.
\subsubsection{Ansible}
Ansible is an open-source piece of software from Red Hat that is described as a configuration management tool, where a form of state description can be written, and then verified through the use of Ansible \citep{powerofansible}. Whilst Ansible does not provide the same level of automation as \gls{sdn}, it does provide a way to automate the configuration of network devices, and can be used to automate the deployment of new devices. The network configuration must be designed and built before Ansible is used to push configuration to the devices. Ansible does help solve issues of scale and complexity, as it can be used to push configuration to multiple devices at once, and can be used to automate the deployment of new devices.

\subsubsection{Chef}
Chef is another open-source configuration management tool that is similar to Ansible. Chef can easily handle up to 10000 nodes from a single chef server \citep{sabharwal2014automation}, however, it is designed to use an agent which must be installed upon the device to be automated which adds another layer of complexity. It also requires the design of configuration and is useful only for deployment and preventing configuration drift.

\section{Developing Software To Interface With SDN}
\label{litreview:developing}
\gls{sdn} already provides the facilities to programmatically interact with the network through the use of \gls{rest} \gls{api}s. Most solutions provide a 'northbound' \gls{api} which can be used by developers to interact with the policy defined in the \gls{sdn} controller. Any changes made via this \gls{api} will then be propagated down to the network devices, with no interaction with the actual devices themselves required. The Northbound \gls{api} allows developers to focus on controlling the \gls{sdn} instead of worrying about sending actual commands down to the network devices \citep{7899569}. There is no standard to northbound \gls{api}s and they are heavily vendor specific \citep{7502469}, this means that an application is only able to support one \gls{sdn} platform, unless compatibility for multiple is explicitly developed and accounted for. This means that the correct platform should be chosen in advance of a solution being developed.


\section{Conclusion}
Software Defined Networking uses a modern approach to improve the efficiency and scalability of conventional networking. Through the use of centralised control and administration, network administrators no longer have to develop such advanced scripts or have as large a workforce to manage and maintain large networks, that can often have thousands of devices. Declarative \gls{sdn} solutions seem to be the way forward, as they take all of the advantages of having centralised control and administration, but don't place as heavy a load on the centralised controller when it comes to influencing packet flow and routing decisions. The ability to easily develop software to control network connectivity, without any of the fuss of integrating and managing any devices that provide physical connectivity is a big bonus and will be imperative to the success of the project. As the solution is highly specific to a testing environment, there are currently no alternative solutions that provide the exact feature set that is desired.

%%%%%%%%%%%%%% REFERENCES SECTION
\newpage
\phantomsection
\addcontentsline{toc}{chapter}{References}
% In order to try and get a consistent format I copy and paste the INSPIRE bibtex code into my bibtex file.
\printbibliography
%%%%%%%%%%%% Include your appendices here
\end{document}